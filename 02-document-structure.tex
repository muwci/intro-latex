\documentclass{article}
\usepackage{fancyhdr}


\title{Typesetting}
\author{Wikipedia}
\date{\today}

\pagestyle{fancy}
\fancyhf{}
\rhead{Wikipedia}
\lhead{Typesetting}
\cfoot{\thepage}

\begin{document}

\maketitle
\tableofcontents

Typesetting is the composition of text by means of arranging physical types[1] or the digital equivalents. Stored letters and other symbols (called sorts in mechanical systems and glyphs in digital systems) are retrieved and ordered according to a language's orthography for visual display. Typesetting requires the prior process of designing a font (which is widely but erroneously confused with and substituted for typeface). One significant effect of typesetting was that authorship of works could be spotted more easily; making it difficult for copiers who have not gained permission.[2]

\section{Pre-Digital era}

\subsection{Manual typesetting}

During much of the letterpress era, movable type was composed by hand for each page. Cast metal sorts were composed into words, then lines, then paragraphs, then pages of text and tightly bound together to make up a form, with all letter faces exactly the same “height to paper”, creating an even surface of type. The form was placed in a press, inked, and an impression made on paper.

During typesetting, individual sorts are picked from a type case with the right hand, and set into a composing stick held in the left hand from left to right, and as viewed by the setter upside down. As seen in the photo of the composing stick, a lower case 'q' looks like a 'd', a lower case 'b' looks like a 'p', a lower case 'p' looks like a 'b' and a lower case 'd' looks like a 'q'. This is reputed to be the origin of the expression "mind your p's and q's". It might just as easily have been “mind your b's and d's”.

The diagram at right illustrates a cast metal sort: a face, b body or shank, c point size, 1 shoulder, 2 nick, 3 groove, 4 foot. Wooden printing sorts were in use for centuries in combination with metal type. Not shown, and more the concern of the casterman, is the “set”, or width of each sort. Set width, like body size, is measured in points.

In order to extend the working life of type, and to account for the finite sorts in a case of type, copies of forms were cast when anticipating subsequent printings of a text, freeing the costly type for other work. This was particularly prevalent in book and newspaper work where rotary presses required type forms to wrap an impression cylinder rather than set in the bed of a press. In this process, called stereotyping, the entire form is pressed into a fine matrix such as plaster of Paris or papier mâché called a flong to create a positive, from which the stereotype form was electrotyped, cast of type metal.

Advances such as the typewriter and computer would push the state of the art even farther ahead. Still, hand composition and letterpress printing have not fallen completely out of use, and since the introduction of digital typesetting, it has seen a revival as an artisanal pursuit. However, it is a very small niche within the larger typesetting market.

\subsection{Hot metal typesetting}

The time and effort required to manually compose the text led to several efforts in the 19th century to produce mechanical typesetting. While some, such as the Paige compositor, met with limited success, by the end of the 19th century, several methods had been devised whereby an operator working a keyboard or other devices could produce the desired text. Most of the successful systems involved the in-house casting of the type to be used, hence are termed "hot metal" typesetting. The Linotype machine, invented in 1884, used a keyboard to assemble the casting matrices, and cast an entire line of type at a time (hence its name). In the Monotype System, a keyboard was used to punch a paper tape, which was then fed to control a casting machine. The Ludlow Typograph involved hand-set matrices, but otherwise used hot metal. By the early 20th century, the various systems were nearly universal in large newspapers and publishing houses.

\subsection{Phototypesetting}

Phototypesetting or "cold type" systems first appeared in the early 1960s and rapidly displaced continuous casting machines. These devices consisted of glass disks (one per font) that spun in front of a light source to selectively expose characters onto light-sensitive paper. Originally they were driven by pre-punched paper tapes. Later they were hooked up to computer front ends.

One of the earliest electronic photocomposition systems was introduced by Fairchild Semiconductor. The typesetter typed a line of text on a Fairchild keyboard that had no display. To verify correct content of the line it was typed a second time. If the two lines were identical a bell rang and the machine produced a punched paper tape corresponding to the text. With the completion of a block of lines the typesetter fed the corresponding paper tapes into a phototypesetting device that mechanically set type outlines printed on glass sheets into place for exposure onto a negative film. Photosensitive paper was exposed to light through the negative film, resulting in a column of black type on white paper, or a galley. The galley was then cut up and used to create a mechanical drawing or paste up of a whole page. A large film negative of the page is shot and used to make plates for offset printing.

\section{Digital era}

The next generation of phototypesetting machines to emerge were those that generated characters on a cathode ray tube. Typical of the type were the Alphanumeric APS2 (1963),[3] IBM 2680 (1967), I.I.I. VideoComp (1973?), Autologic APS5 (1975),[4] and Linotron 202 (1978).[5] These machines were the mainstay of phototypesetting for much of the 1970s and 1980s. Such machines could be "driven online" by a computer front-end system or took their data from magnetic tape. Type fonts were stored digitally on conventional magnetic disk drives.

Computers excel at automatically typesetting and correcting documents.[6] Character-by-character, computer-aided phototypesetting was, in turn, rapidly rendered obsolete in the 1980s by fully digital systems employing a raster image processor to render an entire page to a single high-resolution digital image, now known as imagesetting.

The first commercially successful laser imagesetter, able to make use of a raster image processor was the Monotype Lasercomp. ECRM, Compugraphic (later purchased by Agfa) and others rapidly followed suit with machines of their own.

Early minicomputer-based typesetting software introduced in the 1970s and early 1980s, such as Datalogics Pager, Penta, Atex, Miles 33, Xyvision, troff from Bell Labs, and IBM's Script product with CRT terminals, were better able to drive these electromechanical devices, and used text markup languages to describe type and other page formatting information. The descendants of these text markup languages include SGML, XML and HTML.

The minicomputer systems output columns of text on film for paste-up and eventually produced entire pages and signatures of 4, 8, 16 or more pages using imposition software on devices such as the Israeli-made Scitex Dolev. The data stream used by these systems to drive page layout on printers and imagesetters, often proprietary or specific to a manufacturer or device, drove development of generalized printer control languages, such as Adobe Systems' PostScript and Hewlett-Packard's PCL.

Before the 1980s, practically all typesetting for publishers and advertisers was performed by specialist typesetting companies. These companies performed keyboarding, editing and production of paper or film output, and formed a large component of the graphic arts industry. In the United States, these companies were located in rural Pennsylvania, New England or the Midwest, where labor was cheap and paper was produced nearby, but still within a few hours' travel time of the major publishing centers.

In 1985, desktop publishing became available, starting with the Apple Macintosh, Aldus PageMaker (and later QuarkXPress) and PostScript. Improvements in software and hardware, and rapidly lowering costs, popularized desktop publishing and enabled very fine control of typeset results much less expensively than the minicomputer dedicated systems. At the same time, word processing systems, such as Wang and WordPerfect, revolutionized office documents. They did not, however, have the typographic ability or flexibility required for complicated book layout, graphics, mathematics, or advanced hyphenation and justification rules (H and J).

By the year 2000, this industry segment had shrunk because publishers were now capable of integrating typesetting and graphic design on their own in-house computers. Many found the cost of maintaining high standards of typographic design and technical skill made it more economical to outsource to freelancers and graphic design specialists.

The availability of cheap or free fonts made the conversion to do-it-yourself easier, but also opened up a gap between skilled designers and amateurs. The advent of PostScript, supplemented by the PDF file format, provided a universal method of proofing designs and layouts, readable on major computers and operating systems.


\end{document}
