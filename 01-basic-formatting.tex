\documentclass{article}

\title{On the Comparison of Braids}
\author{Abhik Pal}
\date{November 2015}

\begin{document}
\maketitle

\textbf{LaTeX} (a shortening of \emph{Lamport TeX}) is a \underline{document preparation system}. When writing, the writer uses plain text as opposed to formatted text, as in \textsc{wysiwyg} word
processors like Microsoft Word or LibreOffice Writer. The writer uses markup tagging conventions
to define the general structure of a document (such as article, book, and letter), to stylise text
throughout a document (such as bold and italic), and to add citations and cross-references. A TeX
distribution such as TeX Live or MikTeX is used to produce an output file (such as PDF or DVI)
suitable for printing or digital distribution. Within the typesetting system, its name is stylised
as \LaTeX.

\begin{enumerate}
    \item \textbf{\emph{Original author(s)}}: Leslie Lamport
    \item \textbf{\emph{Initial release}}: 1985; 31 years ago
    \item \textbf{\emph{Repository}}: \texttt{latex-project.org/svnroot/experimental/trunk/}
    \item \textbf{\emph{Type}}: Typesetting
    \item \textbf{\emph{License}}: LaTeX Project Public License (LPPL)
    \item \textbf{\emph{Website}}: \texttt{www.latex-project.org}
\end{enumerate}

\begin{itemize}
    \item \textbf{\emph{Filename extension}}: \texttt{.tex}
    \item \textbf{\emph{Internet media type}}: application/x-latex
    \item \textbf{\emph{Latest release}}: LaTeX \emph{2e} (1994)
    \item \textbf{\emph{Type of format}}: Document file format
\end{itemize}

\end{document}
